\section{Auswertung}
\subsection{Analyse}
Die gemessenen Daten des Beschleunigungssensors werden in $\dfrac{m}{s^{2}}$ 
dargestellt, die Daten des Gyroskops in $\dfrac{\circ}{s}$.

Die Analyse wird durch zwei Faktoren erschwert:
\begin{enumerate}
	\item Die Sensoren sind sehr empfindlich
	\item Baulich bedingt dreht sich das Ei sehr leicht um die vertikale Achse, wodurch eine kurze Beschleunigung auf eine konstante Geschwindigkeit nicht als ein einzelner Ausschlag, sondern als eine Kombination verschiedener dargestellt wird und so kaum erkennbar ist.
\end{enumerate}
Andererseits ist oben genannte Rotation um nur eine Achse sehr leicht in den Daten zu erkennen.

tbd,  Gütemaß

\subsection{Korrelation}
%mit neuer Aufgabe, Erkennung von Modulen


%TODO Abschluss:
%  Zusammenfassung und Fazit
\subsection{Evaluation der Methode}
% Methode: Systemoptimierung durch instrumentiertes Schüttgut
 
- Gewonnen Daten sind sehr inkonsistent
- Noch keine optimalen Filterparameter gefunden
- Mikrocontroller sind störanfällig (Lötstellen, Taktung des Prozessors, BLE-Verbindung)

- noch einige Erweiterungen, damit Segmente erkennbar

FAZIT

\subsection{Ausblick}

%Punkte aus Präsentation
Noch viele Erweiterungen/Änderungen möglich in

- Mikrocontroller: Datenlogging, andere Funkmodule, parallele Messungen (mehrere Kapseln)
- Android App: Echtzeitanzeige, Datenspeicherung, Möglichkeit, Daten direkt an Laptop zu übertragen und auszuwerten
- Client: Anzeige in 3D, UI-Steuerung, Automatisches Finden von ähnlichen Segmenten (Klassifikator)

