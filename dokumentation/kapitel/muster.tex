\section{Musterdoc aus Vorlage - AM ENDE LÖSCHEN, sowie auch LiteraturVerz, falls keine genutzt}

\subsection{Grundlagen des Filters XY}
Vektoren und Matrizen
%
\begin{equation*}
	\vec{x}, \mat{A}
\end{equation*}
%
Mengenzeichen
%
\begin{equation*}
	\IR, \IN
\end{equation*}
%
Zufallsvariablen, etc...
%
\begin{equation*}
	\rv{y}, \rvv{z},
	\Var, \E, \Cov
\end{equation*}
%
Bitte nur Gleichungen nummerieren, auf die sich auch später bezogen wird
%
\begin{equation}
	a = b +c \enspace .
	\label{eq:NameDergleichung}
\end{equation}
%
Laut (\ref{eq:NameDergleichung}) ist $a=b+c$.

Mehrzeiliger Formelsatz mit \emph{align}
%
\begin{align*}
	a &= b + c \enspace ,\\
	a_{ij} &= b_{ij} + c_{ij} \enspace .
\end{align*}
%
oder mit \emph{multline}
%
\begin{multline*}
	a_{2343443} = \\
	b + c + \frac{3464421}{32455767567567575677} 
	\cdot \left( b_{ij} + c_{ij} \right)
	\cdot \int_{x=55}^{88} x^{67823+x} \frac{x}{32455767567567575677} \text{d}x
	\enspace .
\end{multline*}
%

%So werden Bilder eingebunden (als pdf, jpg oder png)
%\begin{figure}[ht]
%  \centering
%  \caption{Hier kommen weitere Erklärungen zum Bild}
% \label{fig:autorname_bild1}
%\end{figure}
%
% Auf diese Abbildung wird dann mit Abb. \ref{fig:autorname_bild1} verwiesen.