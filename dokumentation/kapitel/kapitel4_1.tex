\section{Datenmessung}

\subsection{Erkenntnisse aus dem ersten Testlauf}
Nach dem ersten Testlauf mit der Android-App und dem Bauteil konnten neue wichtige Erkenntnisse gewonnen werden.
Zwischen der App und dem Mikrocontroller kam es zu häufigen Verbindungsabbrüchen. In der App gibt es noch einige Unstimmigkeiten, die behoben werden sollten, damit verwertbare Daten aufgezeichnet werden können. 

\begin{itemize}
	\item Es muss einen fortlaufenden Zähler vom Mikrocontroller aus geben, damit nach einem Verbindungsabbruch und -wiederaufbau der Zähler mitten im Durchlauf nicht auch wieder bei 0 beginnt
	\item Beheben eines Bugs im beim Datenaufzeichnen nach dem ersten Verbindungsabbruch
	\item Der Zugriff auf die UI-Elemente ist zu hoch. Die Ausgabe auf die UI ist langsamer, als die eingehenden Daten, weshalb die App ab vielen Ausgaben eingefrohren ist. Ab dann konnten bisher empfangene Daten auch nicht mehr gespeichert werden
	\item Das Speichern sollte performanter sein
	\item Wie lässt sich ein Rundendurchlauf bzw einzelne markante Punkte in der Anlage währen der Datenaufzeichnung markieren, um nachträglich die analysierten Daten evaluieren zu lassen?
	\item Bluetooth funktioniert über eine Reichweite zwischen 5-10 Metern. Bei zu vielen Verbindungsabbrüchen muss mit dem Empfangsgeräte an der Anlage mitgelaufen werden, da die Wände der Anlage etwas zu abschirmend sind. Eventuell gibt es eine Möglichkeit mehrere BLE-Empfänger an der Anlage zu positionieren
	\item Zum Schutz des Bauteils wurde die Kapsel in Polsterfolie eingepackt. Dadurch wurde das Bauteil langsamer durch die Anlage befördert als die Steine
	\item Ein Akku mit 100 mAh hält doch länger als anfangs angenommen
\end{itemize}

\subsection{Hauptlauf}

Nach Optimierung der Android App konnten wieder Daten gemessen werden. Die Bluetooth-Verbindungsaufbau wurde verändert, sodass die Verbindung nun viel stabiler lief. Pro Rundendurchgang wurde ein interner Zähler der App auf 0 gesetzt, um Runden im Datensatz identifizieren zu können. Zusätzlich wurde an immer gleichen Punkte in der Anlage die empfangen Daten zwischen gespeichert. Da damit eine neue Datei angelegt wurde, können auch einzelne Anlagenabschnitte ungefähr zugeordnet werden.

Die Kapsel wurde bei den ersten Messrunden in Polsterfolie eingepackt. Da dadurch die Kapsel auf der Anlage langsamer als die Steine befördert wurde, wurden an die Verpackung noch drei Flügel angebracht, damit Steine die Kapsel besser mitnehmen können. Dadurch war die Kapsel ebenso schnell wie die Steine und lag auch auf den einzelnen Beförderungsmodulen sehr stabil (wenig Orientierungsänderungen). Zum Abschluss wurde die Kapsel ohne Verpackungsmaterial durch die Anlage geschickt. Die Kapsel war hier ebenso schnell wie die Steine, allerdings war hier zu beobachten, dass die Kapsel sich sehr viel drehte. 

In den letzten Dateien kam es zu Testfehlern, indem keine aktuellen Daten vom Sensormodul mehr gesendet wurden.

%TODO Bilder von den unterschiedlichen Kapselmessungen